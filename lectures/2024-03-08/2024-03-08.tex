%!TEX root  = 2024-03-08.tex
\documentclass[../../main/main.tex]{subfiles}
\begin{document}
\subsection {Lecture 1 - Asymptotic Notation}

\begin{theorem}
	Upper bounds $T(n) = O(f(n))$.

	if exists constants $c > 0$ and $n_{0}$ such that $\forall n\ge n_{0}$, $T(n) \le c \cdot f(n)$
\end{theorem}

\begin{theorem}
	Lower bounds $T(n) = \Omega(f(n))$

	if exists constants $c > 0$ and $n_{0}$ such that $\forall n\ge n_{0}$, $T(n) \ge  c \cdot f(n)$
\end{theorem}
\begin{theorem}
	Tight bounds $T(n) = \Theta(f(n))$

	if $T(n) = O(f(n))$ and  $T(n) = \Omega(f(n))$
\end{theorem}

constant $ 9999^{9999^{9999}}<$ logarithmic $\log^{10} n < $ polynomial $n^{0.1} < n \log n < n^2$ $<$ exponential $2^{n}$

\begin{enumerate}
	\item $9999^{99999^{9999}} = \Theta(1) = O(log(log(n)))$
	\item $log(log(n)) &= O(log n)$
	\item $n^{100} &= O(2^{n})$
	      Let $n^{100} = c \cdot 2^{n}$ s.t. for $ n \ge  n_{0} n^{100} \le c\cdot 2^{n}$

	      \begin{align*}
		      100 * \log(n) & = c \cdot n \log (2)      \\
		      n             & = \frac{100}{c} * \log(n) \\
		      .\end{align*}
	      $\therefore \forall c, n^{100} = O(2^{n}) $
\end{enumerate}

\begin{theorem} Common expressions
	\begin{enumerate}
		\item $max(f(n), g(n)) = \Theta(f(n) + g(n))$
		\item $\log\sqrt{n} = \Omega(\sqrt{\log(n)} ) $
		\item $\log(2^{n}) = \Theta(\log(3^{n}))$
		\item $\sum^{n}_{i=1} \frac{1}{i} = \Theta(\log n)$
	\end{enumerate}
\end{theorem}
\newpage
\subsubsection {Lecture 2 - Running time of sorting}
\subsubsection{Selection sort}
\begin{algorithm}[H]
	\caption{Selection sort}\label{alg:cap}
	\begin{algorithmic}
		\For{$i \gets 1$ to  $n$}
		\For{$j \gets i+1$ to $n$}
		\If{$A[j] < A[i]$}
		\State swap $A[j], A[i]$
		\EndIf
		\EndFor

		\EndFor
	\end{algorithmic}
\end{algorithm}
number of comparisons : $\sum^{n}_{i=1} (n-i) = \sum^{n-1}_{i=1}  = \frac{n(n-1)}{2}$

\begin{theorem}
	Correctness of selection sort

	Prove by induction. Assume the algorithnm sorts every array of size n-1 correctly.
	\begin{enumerate}
		\item It first pusts the smallest item in $A[1]$
		\item then runs selection sort on  $A[2\ldots n]$ (by induction this is correct)
		\item since $A[1]$ is smaller than every other items, the array is sorted
	\end{enumerate}
\end{theorem}

\subsubsection{Insertion sort}
\begin{algorithm}[H]
	\caption{Insertion sort}
	\begin{algorithmic}
		\For{$i \gets 2$ to  $n$}
		\State $j \gets i-1$
		\While{$j \ge 1$ and $A[j] > A[j+1]$}
		\State \texttt{swap $A[j]$ and $A[j+1]$}
		\State \texttt{$j \gets j - 1$}
		\EndWhile
		\EndFor
	\end{algorithmic}
\end{algorithm}
\begin{enumerate}
	\item	      number of comparisons : $\sum^{n}_{i=1} (i-1) &= \sum^{n-1}_{k=1} k &= \frac{n(n-1)}{2}$
	\item average case analysis: only half of the keys are compared
	\item best case: on sorted data, takes $O(n)$ time

\end{enumerate}

\begin{theorem}
	Comparison of running time
	\begin{enumerate}
		\item $O(\log n)$ U $\Theta(2^{\log_{2}\log_{2} n} )$
		\item $O(n^{4})$ U $O(n^{3})$
	\end{enumerate}
\end{theorem}
\begin{exercise}
	Prove that $\log(n!) = \Theta(n\log n)$
	\begin{enumerate}
		\item first, prove $\log(n!) = O(n\log n)$
		      \begin{align*}
			      \log(n!) & =  \sum^{n}_{i=1} \log(i)   \\
			               & \le \sum^{n}_{i=1}  \log(n) \\
			               & = O(n\log n)
			      .\end{align*}
		\item then, prove $\log(n!) = \Omega(n\log n)$
		      \begin{align*}
			      \log(n!) & =  \sum^{n}_{i=1} \log(i)                       \\
			               & \ge \sum^{n}_{i=\frac{n}{2}}  \log(i)           \\
			               & \ge \sum^{n}_{i=\frac{n}{2}}  \log(\frac{n}{2}) \\
			               & = \frac{n}{2}\log \frac{n}{2}                   \\
			               & = \frac{n}{2}(\log n - \log2)                   \\
			               & = \Omega(n\log n)
			      .\end{align*}
		      $\therefore \log(n!) = \Theta(n\log n)$
	\end{enumerate}
\end{exercise}
\newpage
\subsubsection{Lecture 3 - Divide and conquer}
\begin{theorem}
	Run time analysis of Binary search

	\begin{align*}
		T(n) & = T(\frac{n}{2}) + 2  \; \text{if $n > 1$, with $T(1) = 1$} \\
		     & = T(\frac{n}{2^2} + 2) + 2                                  \\
		     & = \ldots                                                    \\
		     & = T(\frac{n}{2^{\log_{2}n}})  + 2\log_2n                    \\
		     & = 1 + 2\log_{2}n
		.\end{align*}
\end{theorem}

Examples
\begin{enumerate}
	\item Rotated sorted array
	\item Find the last 0
\end{enumerate}

\begin{algorithm}[H]
	\caption{Tower of hanoi(n, peg1, peg2, peg3)}
	\begin{algorithmic}[1]
		\If{n=0}
		\texttt{do something}
		\Else
		\State \texttt{Tower of hanoi(n-1, peg1, peg2, peg3)} \Comment{T(n-1)}
		\State \texttt{move the only disc from peg 1 to peg 3} \Comment{T(1)}

		\State\texttt{Tower of hanoi(n-1, peg2, peg1, peg3)} \Comment{T(n-1)}
		\EndIf
	\end{algorithmic}
\end{algorithm}

\begin{theorem}
	Recurrence of Tower of hanoi:
	\begin{align*}
		T(n) & = 2T(n-1) + 1                                              \\
		     & = 2(2T(n-2) + 1) + 1                                       \\
		     & = \ldots                                                   \\
		     & = 2^{n-1}T(1) +  2^{n-2} + 2^{n-3}  + \ldots + 2^2 + 2 + 1 \\
		     & = 2^{n} - 1
		.\end{align*}

\end{theorem}
\begin{algorithm}[H]
	\caption{Merge sort(A, l, r)}
	\begin{algorithmic}[1]
		\If{$l = r$}
		\texttt{return}
		\Else
		\State mid $\gets \frac{l+r}{2}$
		\State \texttt{Merge Sort(A, l, mid)} \Comment{T($\frac{n}{2}$)}
		\State \texttt{Merge Sort(A, mid+1, r)} \Comment{T($\frac{n}{2}$)}
		\State \texttt{Merge(A, l, mid, r)} \Comment{O(n)}
		\EndIf
	\end{algorithmic}
\end{algorithm}
\begin{theorem}
	Analysis of merge sort
	\begin{align*}
		T(n) & \le  T(\left\lfloor \frac{n}{2} \right\rfloor) + T(\left\lceil \frac{n}{2} \right\rceil ) + O(n)                                 \\
		     & = 2^{i}T(\frac{n}{2^{i}}) + in                                                                   &  & \text{where $i = \log_2n$} \\
		     & = n\log_2n + n
		.\end{align*}
\end{theorem}

\end{document}
