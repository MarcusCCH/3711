%TEX root = 2024-03-10.tex
\documentclass[../../main/main.tex]{subfiles}
\begin{document}
\subsection{Lecture 7 - Heapsort}
priority queue
\begin{enumerate}
  \item insert min in $O(\log n)$
  \item extract min in $O(\log n)$
\end{enumerate}

\begin{definition}
  min-heap
  \begin{enumerate}
    \item children is as large as the parent
    \item add new element to the next available position at the lowest level 
  \end{enumerate}
\end{definition}

% \usepackage{algorithm,algorithmicx,algpseudocode}
\begin{algorithm}
  \floatname{algorithm}{Algorithm}
  \algrenewcommand\algorithmicrequire{\textbf{Input: }}
  \algrenewcommand\algorithmicensure{\textbf{Output: }}
  \caption{Extract min}\label{alg:}
  \begin{algorithmic}[1]
    \Require $A$
    \Ensure $min$
    \State $min \gets A[i]$
    \State \Call{assign}{A[1],A[n]}
    \State \Call{assign}{A[n],$\infty$}
    \State $j \gets 1$
    \State $l \gets A[2j]; r \gets A[2j+1]$
    \While{$A[j] > min(A[l],A[r])$} \Comment{heapify}
    \If{$l > r$}
    \State \Call{swap}{A[j],A[2j]}
    \State $j \gets 2j$
    \Else
    \State \Call{swap}{A[j],A[2j+1]}
    \State $j \gets 2j+1$
    \EndIf
    \State $l \gets A[2j]; r \gets A[2j+1]$

    \State \textbf{return} $min$
  \end{algorithmic}
\end{algorithm}

\begin{definition}
  heap-sort
  \newline
  build a binary heap and insert elements, extract min in $O(n\log n)$
\end{definition}

\begin{remark}
  How to support decrease-key in $O(\log n)$?
\end{remark}
\subsection{Lecture 8 - Sorting in linear time}
\begin{claim}
  Properties of sorting
  \begin{enumerate}
    \item Comparison based sorting have $\Omega(n\log n)$
    \item $h\ge \log n$
  \end{enumerate}
\end{claim}
\subsubsection{Counting sort}
% \usepackage{algorithm,algorithmicx,algpseudocode}
\begin{algorithm}[H]
  \floatname{algorithm}{Algorithm}
  \algrenewcommand\algorithmicrequire{\textbf{Input: }}
  \algrenewcommand\algorithmicensure{\textbf{Output: }}
  \caption{Counting sort}\label{alg:}
  \begin{algorithmic}[1]
    \Require $A,k$
    \Ensure $B$
    \State $C[0 \dots k] \gets 0$
    \For{$i \gets 1$ to $n$}
    \State $C[A[i]] \gets C[A[i]]+1$
    \EndFor

    \For{$i \gets 1$ to $k$}
    \State $C[i] \gets C[i]+C[i-1]$
    \EndFor

    \For{$i \gets n$ to $1$}
    \State $idx \gets C[A[i]]$
    \State $B[idx] \gets A[i]$
    \State $C[A[i]] \gets C[A[i]]-1$
    \EndFor
    \State \textbf{return} $B$
  \end{algorithmic}
\end{algorithm}

\subsubsection{Radix sort}
\begin{definition}[Radix sort]
  use counting sort to sort digits from least significant to most significant
\end{definition}

\pvocab{Proof of correctness.}

Assume we already sorted the lower digits $0\ldots k-1$. We need to sort the most significant digit. WLOG, suppose we have a digit d, by definition, all numbers
with d as the k-th digit are sorted but not necessarily in continuous positions. Counting sort simply puts numbers with the same k-th digit in the continuous order. 

\newline
\pvocab{Run-time analysis.}

Counting sort is run $d$ times and total run time is  $\Theta(d(n+k))}$

\newline
\pvocab{Decision model on list merging.}
\begin{enumerate}
  \item number of leafs = $k!$
  \item number of possible permutations =  $(k!)^{\frac{n}{k}}$
  \item height = $\log ((k!)^{\frac{n}{k}})$ = $\Theta(n\log k)$
\end{enumerate}

\end{document}

